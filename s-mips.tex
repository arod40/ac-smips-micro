\documentclass[letterpaper,11pt]{scrartcl}
\usepackage{tabularx}
\usepackage{tabulary}
\usepackage{graphicx}
\usepackage{hyperref}
\usepackage{multicol}
\usepackage[margin=0.7in]{geometry}
\usepackage[spanish]{babel}
\usepackage[utf8]{inputenc}

\hypersetup{colorlinks,urlcolor=blue,citecolor=blue,linkcolor=black}

\title{Proyecto de Arquitectura de Computadoras \\ \small{Curso 2016-2017}}
\date{}
\author{}

\begin{document}

\pagestyle{empty}

\setlength{\parskip}{0.1em}

\maketitle
\thispagestyle{empty}

\vspace{-2.5cm}
\begin{center}
 \huge{\textbf{S-MIPS}}
\end{center}
\vspace{0.5cm}

\begin{center}
\includegraphics[height=1.5in]{./mips.jpg}\\
Procesador S-MIPS
\end{center}

\section*{Introducción}

El proyecto consiste en diseñar en LogiSim un procesador que implemente la arquitectura 
de juegos de instrucciones S-MIPS (Simplified-MIPS).

S-MIPS es una arquitectura de 32 bits.  El procesador tiene 32 registros de
32 bits, nombrados R0 a R31, de propósito general, de ellos R0 siempre tiene el valor constante 0
independientemente de las operaciones que se realizen sobre él y R31 (SP)
que actúa como puntero de la pila (stack pointer). Cuenta con dos registros adicionales
\textbf{Hi} y  \textbf{Lo} donde se almacena el resultado de la división y la multiplicación.

El procesador debe implementarse en el módulo \textbf{S-MIPS} del circuito \textbf{S-MIPS Board} que se brinda.
No debe modificar el circuto \textbf{RAM} ni el circuto \textbf{S-MIPS Board}. Durante la evaluación se utilizarán
estos circuitos tal y como se les entregó.

\section*{Instrucciones y su formato}

\subsection*{Notas generales:}
\begin{itemize}
    \item{ \textbf{R\textsubscript{s}}, \textbf{R\textsubscript{t}} y \textbf{R\textsubscript{d}} especifican registros de propósito general.}
    \item{ Un elemento entre corchetes (\textbf{[ ]}) indica "el contenido de". Por ejemplo $[R3] + [R22]$ se refiere a 
    la suma de los valores almacenados en los registros R3 y R22.}
    \item{ \textbf{[PC]} especifica la dirección de la instrucción en ejecución. Por ejemplo saltar a la próxima instrucción es 
    $[PC] = [PC] + 4$ }
    \item \textbf{I} se refiere a los bits de la instrucción y el subindice indica a cuales de estos bits se refiere. $[I\textsubscript{15..0}]$ 
    se refiere al contenido de los 16 primeros bits de la instrucción, que en el caso de las instrucciones de tipo I es la constante.
    \item \textbf{ $\Vert$ }  indica la concatenación de bits.
    \item Los sobreíndices indican la repetición de un valor binario. $0\textsuperscript{7}$ se refiere a: \verb|0000000|.
    \item \textbf{M\{I\}} se refiere a los 32 bits almacenados en la memoria RAM en la dirección divisible por 4
    más cercana al valor de I. ($I-I\%4$).
\end{itemize}

Las instrucciones de S-MIPS tienen una longitud constante de 32 bits. Hay 3 formatos de
instrucciones distintos:

\begin{table*}[!htbp]
\begin{small}
\begin{verbatim}
                                     6       5      5      5      5        6
                                +---------+------+------+------+-----+-------------+
           Formato de tipo R:   | Op-code |  Rs  |  Rt  |  Rd  |  X  |  Funct-code |
                                +---------+------+------+------+-----+-------------+
                                     6       5      5               16
                                +---------+------+------+--------------------------+
           Formato de tipo I:   | Op-code |  Rs  |  Rt  | constante(complemento 2) |
                                +---------+------+------+--------------------------+
                                     6                       26
                                +---------+----------------------------------------+
           Formato de tipo J:   | Op-code |            destino_salto               |
                                +---------+----------------------------------------+
                                |                                                  |
                            bit 31                                             bit 0

\end{verbatim} 
\end{small}
\end{table*}

\section*{Instrucciones}
% ---------------------------------- nop --------------------------------------

\subsubsection*{No operation: \textbf{nop}}

\begin{verbatim}
                     +--------+-------+-------+-------+-------+--------+ 
Formato de tipo R:   | 000000 | 00000 | 00000 | 00000 | 00000 | 000000 | 
                     +--------+-------+-------+-------+-------+--------+ 
\end{verbatim}

Efectos de la instrucción: 

$ PC \leftarrow [PC] + 4 $

Ensamblador: $nop$

% ---------------------------------- add --------------------------------------

\subsubsection*{Addition: \textbf{add}}

\begin{verbatim}
                     +--------+-------+-------+-------+-------+--------+ 
Formato de tipo R:   | 000000 |   Rs  |   Rt  |   Rd  | 00000 | 100000 | 
                     +--------+-------+-------+-------+-------+--------+ 
\end{verbatim}

Efectos de la instrucción: 

$R_{d} \leftarrow [R_{s}] + [R_{t}]$

$PC \leftarrow [PC] + 4$

Ensamblador: $add\hspace{5pt}R_{d}, R_{s}, R_{t}$

% ---------------------------------- sub --------------------------------------

\subsubsection*{Subtract: \textbf{sub}}

\begin{verbatim}
                     +--------+-------+-------+-------+-------+--------+ 
Formato de tipo R:   | 000000 |   Rs  |   Rt  |   Rd  | 00000 | 100010 | 
                     +--------+-------+-------+-------+-------+--------+ 
\end{verbatim}

Efectos de la instrucción: 

$R_{d} \leftarrow [R_{s}] - [R_{t}]$

$PC \leftarrow [PC] + 4$

Ensamblador: $sub\hspace{5pt}R_{d}, R_{s}, R_{t}$

% ---------------------------------- mul --------------------------------------

\subsubsection*{Multiply: \textbf{mult}}

\begin{verbatim}
                     +--------+-------+-------+-------+-------+--------+ 
Formato de tipo R:   | 000000 |   Rs  |   Rt  | 00000 | 00000 | 011000 | 
                     +--------+-------+-------+-------+-------+--------+ 
\end{verbatim}

Efectos de la instrucción: 

$Hi || Lo \leftarrow [R_{s}] * [R_{t}]$

$PC \leftarrow [PC] + 4$

Ensamblador: $mult\hspace{5pt}R_{s}, R_{t}$

% ---------------------------------- mulu --------------------------------------

\subsubsection*{Unsigned multiply: \textbf{mulu}}

Idéntica a la instucción \textbf{mult} excepto:

- Funct-code = 011001

- El contenido de $R_{s}$ y $R_{t}$ se considera como enteros sin signo.

% ----------------------------------div --------------------------------------

\subsubsection*{Divide: \textbf{div}}

\begin{verbatim}
                     +--------+-------+-------+-------+-------+--------+ 
Formato de tipo R:   | 000000 |   Rs  |   Rt  | 00000 | 00000 | 011010 | 
                     +--------+-------+-------+-------+-------+--------+ 
\end{verbatim}

Efectos de la instrucción: 

$Lo \leftarrow [R_{s}] / [R_{t}]$

$Hi \leftarrow [R_{s}] mod [R_{t}]$

$PC \leftarrow [PC] + 4$

Ensamblador: $div\hspace{5pt}R_{s}, R_{t}$

% ---------------------------------- divu --------------------------------------

\subsubsection*{Unsigned divide: \textbf{divu}}

Idéntica a la instucción \textbf{div} excepto:

- Funct-code = 011011

- El contenido de $R_{s}$ y $R_{t}$ se considera como enteros sin signo.

% ---------------------------------- slt --------------------------------------

\subsubsection*{Set less than: \textbf{slt}}

\begin{verbatim}
                     +--------+-------+-------+-------+-------+--------+ 
Formato de tipo R:   | 000000 |   Rs  |   Rt  |   Rd  | 00000 | 101010 | 
                     +--------+-------+-------+-------+-------+--------+ 
\end{verbatim}

Efectos de la instrucción: 

si   $[R_{s}] < [R_{t}]$ entonces $R_{d} \leftarrow 0^{31} || 1$

sino $R_{d} \leftarrow 0^{32}$

$PC \leftarrow [PC] + 4$

Ensamblador: $slt\hspace{5pt}R_{d}, R_{s}, R_{t}$

% ---------------------------------- sltu --------------------------------------

\subsubsection*{Unsigned set less than: \textbf{sltu}}

Idéntica a la instucción \textbf{slt} excepto:

- Funct-code = 101011

- El contenido de $R_{s}$ y $R_{t}$ se considera como enteros sin signo.

% ---------------------------------- and --------------------------------------

\subsubsection*{Logical and: \textbf{and}}

\begin{verbatim}
                     +--------+-------+-------+-------+-------+--------+ 
Formato de tipo R:   | 000000 |   Rs  |   Rt  |   Rd  | 00000 | 100100 | 
                     +--------+-------+-------+-------+-------+--------+ 
\end{verbatim}

Efectos de la instrucción: 

$R_{d} \leftarrow [R_{s}] AND [R_{t}]$

$PC \leftarrow [PC] + 4$

Ensamblador: $and\hspace{5pt}R_{d}, R_{s}, R_{t}$

% ---------------------------------- or, nor, xor --------------------------------------

\subsubsection*{Logical or, nor, xor: \textbf{or, nor, xor}}

Idénticas a la instucción \textbf{and} excepto:

- Funct-code = 100101 para la instruccion \textbf{or}

- Funct-code = 100111 para la instruccion \textbf{nor}

- Funct-code = 101000 para la instruccion \textbf{xor}

- Se realiza la función lógica apropiada en vez de \textbf{and}.

% ---------------------------------- addi --------------------------------------

\subsubsection*{Addition inmediate: \textbf{addi}}

\begin{verbatim}
                     +---------+------+------+--------------------------+
Formato de tipo I:   | 001000  |  Rs  |  Rt  |        constante         |
                     +---------+------+------+--------------------------+
\end{verbatim}

Efectos de la instrucción: 

$R_{t} \leftarrow [R_{s}] + ( [I_{15}]^{16} || [I_{15..0}] )$

$PC \leftarrow [PC] + 4$

Ensamblador: $addi\hspace{5pt}R_{t}, R_{s}, constante$, por ejemplo: $addi\hspace{5pt}R_{3}, R_{8}, 34$

% ---------------------------------- slti --------------------------------------

\subsubsection*{Set less than immediate: \textbf{slti}}

\begin{verbatim}
                     +--------+-------+-------+------------------------+ 
Formato de tipo I:   | 001010 |   Rs  |   Rt  |       constante        | 
                     +--------+-------+-------+------------------------+ 
\end{verbatim}

Efectos de la instrucción: 

si $[R_{s}] < ([I_{15}]^{16}||[I_{15..0}])$ entonces $R_{t} \leftarrow 0^{31} || 1$

sino $R_{t} \leftarrow 0^{32}$

$PC \leftarrow [PC] + 4$

Ensamblador: $slti\hspace{5pt}R_{t}, R_{s}, constante$

% ---------------------------------- sltiu --------------------------------------

\subsubsection*{Set less than immediate unsigned: \textbf{sltiu}}

Idéntica a la instucción \textbf{slti} excepto:

- Funct-code = 001011

- El contenido de $R_{s}$ y $R_{t}$ se considera como enteros sin signo.


% ---------------------------------- andi --------------------------------------

\subsubsection*{Logical and immediate: \textbf{andi}}

\begin{verbatim}
                     +--------+-------+-------+------------------------+ 
Formato de tipo I:   | 001100 |   Rs  |   Rt  |       constante        | 
                     +--------+-------+-------+------------------------+ 
\end{verbatim}

Efectos de la instrucción: 

$R_{t} \leftarrow [R_{s}]AND (0^{16} || [I_{15..0}])$

$PC \leftarrow [PC] + 4$

Ensamblador: $andi\hspace{5pt}R_{t}, R_{s}, constante$

% ---------------------------------- ori, xori --------------------------------------

\subsubsection*{Logical or immediate \& xor immediate: \textbf{ori, xori}}

Idénticas a la instucción \textbf{andi} excepto:

- Funct-code = 001101 para la instruccion \textbf{ori}

- Funct-code = 001110 para la instruccion \textbf{xori}

- Se realiza la función lógica apropiada en vez del \textbf{and} lógico.

% ---------------------------------- lw --------------------------------------

\subsubsection*{Load word: \textbf{lw}}

\begin{verbatim}
                     +--------+-------+-------+------------------------+ 
Formato de tipo I:   | 100011 |   Rs  |   Rt  |        offset          | 
                     +--------+-------+-------+------------------------+ 
\end{verbatim}

Efectos de la instrucción: 

$R_{t} \leftarrow M\{[R_{s}] + [I_{15}]^{16} || [I_{15..0}] \}$

$PC \leftarrow [PC] + 4$

Ensamblador: $lw\hspace{5pt}R_{t}, offset(R_{s})$ por ejemplo: $lw\hspace{5pt}R_{3}, 16(R_{0})$

% ---------------------------------- sw --------------------------------------

\subsubsection*{Store word: \textbf{sw}}

\begin{verbatim}
                     +--------+-------+-------+------------------------+ 
Formato de tipo I:   | 101011 |   Rs  |   Rt  |        offset          | 
                     +--------+-------+-------+------------------------+ 
\end{verbatim}

Efectos de la instrucción: 

$M\{[R_{s}] + [I_{15}]^{16} || [I_{15..0}] \} \leftarrow [R_{t}] $

$PC \leftarrow [PC] + 4$

Ensamblador: $sw\hspace{5pt}R_{t}, offset(R_{s})$

% ---------------------------------- pop --------------------------------------

\subsubsection*{Pop from stack: \textbf{pop}}

\begin{verbatim}
                     +--------+-------+-------+-------+-------+--------+ 
Formato de tipo R:   | 111000 | 00000 | 00000 |  Rd   | 00000 | 000000 | 
                     +--------+-------+-------+-------+-------+--------+ 
\end{verbatim}

Efectos de la instrucción: 

$R_d \leftarrow M\{[SP]\}$

$SP \leftarrow [SP] + 4$

$PC \leftarrow [PC] + 4$

Ensamblador: $pop\hspace{5pt}R_{d}$

% ---------------------------------- push --------------------------------------

\subsubsection*{Push to stack: \textbf{push}}

\begin{verbatim}
                     +--------+-------+-------+-------+-------+--------+ 
Formato de tipo R:   | 111000 |   Rs  | 00000 | 00000 | 00000 | 000001 | 
                     +--------+-------+-------+-------+-------+--------+ 
\end{verbatim}

Efectos de la instrucción: 

$SP \leftarrow [SP] - 4$

$M\{[SP]\} \leftarrow R_s $

$PC \leftarrow [PC] + 4$

Ensamblador: $push\hspace{5pt}R_{s}$

% ---------------------------------- beq --------------------------------------

\subsubsection*{Branch on equal: \textbf{beq}}

\begin{verbatim}
                     +--------+-------+-------+------------------------+ 
Formato de tipo I:   | 000100 |   Rs  |   Rt  |        offset          | 
                     +--------+-------+-------+------------------------+ 
\end{verbatim}

Efectos de la instrucción: 

si $[R_{s}] == [R_{t}]$ entonces $PC \leftarrow [PC] + 4 + ( [I_{15}]^{14} || [I_{15..0}] || 0^{2} )$ o sea: $PC \leftarrow [PC] + 4 + 4*offset$

sino $PC \leftarrow [PC] + 4$

Ensamblador: $beq\hspace{5pt}R_{s}, R_{t}, offset$

% ---------------------------------- bne --------------------------------------

\subsubsection*{Branch on not equal: \textbf{bne}}

\begin{verbatim}
                     +--------+-------+-------+------------------------+ 
Formato de tipo I:   | 000101 |   Rs  |   Rt  |        offset          | 
                     +--------+-------+-------+------------------------+ 
\end{verbatim}

Efectos de la instrucción: 

si $[R_{s}] <> [R_{t}]$ entonces $PC \leftarrow [PC] + 4 + ( [I_{15}]^{14} || [I_{15..0}] || 0^{2} )$ o sea: $PC \leftarrow [PC] + 4 + 4*offset$

sino $PC \leftarrow [PC] + 4$

Ensamblador: $bne\hspace{5pt}R_{s}, R_{t}, offset$

% ---------------------------------- blez --------------------------------------

\subsubsection*{Branch on less than or equal zero: \textbf{blez}}

\begin{verbatim}
                     +--------+-------+-------+------------------------+ 
Formato de tipo I:   | 000110 |   Rs  |   Rt  |        offset          | 
                     +--------+-------+-------+------------------------+ 
\end{verbatim}

Efectos de la instrucción: 

si $[R_{s}] <= 0$ entonces $PC \leftarrow [PC] + 4 + ( [I_{15}]^{14} || [I_{15..0}] || 0^{2} )$ o sea: $PC \leftarrow [PC] + 4 + 4*offset$

sino $PC \leftarrow [PC] + 4$

Ensamblador: $blez\hspace{5pt}R_{s}, offset$

% ---------------------------------- bgtz --------------------------------------

\subsubsection*{Branch on greater than zero: \textbf{bgtz}}

\begin{verbatim}
                     +--------+-------+-------+------------------------+ 
Formato de tipo I:   | 000111 |   Rs  |   Rt  |        offset          | 
                     +--------+-------+-------+------------------------+ 
\end{verbatim}

Efectos de la instrucción: 

si $[R_{s}] > 0$ entonces $PC \leftarrow [PC] + 4 + ( [I_{15}]^{14} || [I_{15..0}] || 0^{2} )$ o sea: $PC \leftarrow [PC] + 4 + 4*offset$

sino $PC \leftarrow [PC] + 4$

Ensamblador: $bgtz\hspace{5pt}R_{s}, offset$

% ---------------------------------- bltz --------------------------------------

\subsubsection*{Branch on less than zero: \textbf{bltz}}

\begin{verbatim}
                     +--------+-------+-------+------------------------+ 
Formato de tipo I:   | 000001 |   Rs  |   Rt  |        offset          | 
                     +--------+-------+-------+------------------------+ 
\end{verbatim}

Efectos de la instrucción: 

si $[R_{s}] < 0$ entonces $PC \leftarrow [PC] + 4 + ( [I_{15}]^{14} || [I_{15..0}] || 0^{2} )$ o sea: $PC \leftarrow [PC] + 4 + 4*offset$

sino $PC \leftarrow [PC] + 4$

Ensamblador: $bltz\hspace{5pt}R_{s}, offset$

% ---------------------------------- j --------------------------------------

\subsubsection*{Jump: \textbf{j}}

\begin{verbatim}
                     +--------+----------------------------------------+ 
Formato de tipo J:   | 000010 |                destino                 | 
                     +--------+----------------------------------------+ 
\end{verbatim}

Efectos de la instrucción: 

$PC \leftarrow [PC_{31..28}] || [I_{25..0}] || 0^{2} $

Ensamblador: $j\hspace{5pt} destino$

% ---------------------------------- jr --------------------------------------

\subsubsection*{Jump register: \textbf{jr}}

\begin{verbatim}
                     +--------+-------+-------+-------+-------+--------+ 
Formato de tipo R:   | 000000 |   Rs  | 00000 | 00000 | 00000 | 001000 | 
                     +--------+-------+-------+-------+-------+--------+ 
\end{verbatim}

Efectos de la instrucción: 

$PC \leftarrow [R_s]$

Ensamblador: $jr\hspace{5pt}R_{s}$

% ---------------------------------- mfhi --------------------------------------

\subsubsection*{Move from Hi: \textbf{mfhi}}

\begin{verbatim}
                     +--------+-------+-------+-------+-------+--------+ 
Formato de tipo R:   | 000000 | 00000 | 00000 |  Rd   | 00000 | 010000 | 
                     +--------+-------+-------+-------+-------+--------+ 
\end{verbatim}

Efectos de la instrucción: 

$R_d \leftarrow [Hi]$

$PC \leftarrow [PC] + 4$

Ensamblador: $mfhi\hspace{5pt}R_{d}$

% ---------------------------------- mflo --------------------------------------

\subsubsection*{Move from Lo: \textbf{mflo}}

\begin{verbatim}
                     +--------+-------+-------+-------+-------+--------+ 
Formato de tipo R:   | 000000 | 00000 | 00000 |  Rd   | 00000 | 010010 | 
                     +--------+-------+-------+-------+-------+--------+ 
\end{verbatim}

Efectos de la instrucción: 

$R_d \leftarrow [Lo]$

$PC \leftarrow [PC] + 4$

Ensamblador: $mflo\hspace{5pt}R_{d}$

\subsection*{Instrucciones especiales}

Las siguientes instrucciones no son realistas pero son necesarias para simular una entrada de teclado y una terminal
sin complicar la interfaz de simulación.

% ---------------------------------- halt --------------------------------------

\subsubsection*{Halt program: \textbf{halt}}

\begin{verbatim}
                     +--------+-------+-------+-------+-------+--------+ 
Formato de tipo R:   | 111111 | 00000 | 00000 | 00000 | 00000 | 111111 | 
                     +--------+-------+-------+-------+-------+--------+ 
\end{verbatim}

Efectos de la instrucción: 

- Detiene la simulación del programa actual.

Ensamblador: $halt$

% ---------------------------------- tty --------------------------------------

\subsubsection*{Write to terminal: \textbf{tty}}

\begin{verbatim}
                     +--------+-------+-------+-------+-------+--------+ 
Formato de tipo R:   | 111111 |   Rs  | 00000 | 00000 | 00000 | 000001 | 
                     +--------+-------+-------+-------+-------+--------+ 
\end{verbatim}

Efectos de la instrucción: 

Envia un caracter a la pantalla  conectada al procesador (TTY). La pantalla es un circuito síncrono.
Para enviar un caracter, se ponen los 7 bits menos significativos de Rs en la salida TTY DATA del procesador, 
se activa la salida TTY ENABLE y en el próximo ciclo la pantalla mostrará el caracter ASCII correspondiente 
a los 7 bits de TTY DATA.

Ensamblador: $tty\hspace{5pt}R_s$

% ---------------------------------- rnd --------------------------------------

\subsubsection*{Set random: \textbf{rnd}}

\begin{verbatim}
                     +--------+-------+-------+-------+-------+--------+ 
Formato de tipo R:   | 111111 | 00000 | 00000 |   Rd  | 00000 | 000010 | 
                     +--------+-------+-------+-------+-------+--------+ 
\end{verbatim}

Efectos de la instrucción: 

Almacena en Rd un número aleatorio.

Ensamblador: $rnd\hspace{5pt}R_s$

% ---------------------------------- kbd --------------------------------------

\subsubsection*{Set key: \textbf{kbd}}

\begin{verbatim}
                     +--------+-------+-------+-------+-------+--------+ 
Formato de tipo R:   | 111111 | 00000 | 00000 |   Rd  | 00000 | 000100 | 
                     +--------+-------+-------+-------+-------+--------+ 
\end{verbatim}

Efectos de la instrucción: 

Esta instrucción lee un caracter del teclado y lo almacena en Rd.
La entrada KBD AVAILABLE del procesador indica si hay algún caracter
esperando en el buffer del teclado y la entrada KBD DATA, es un número
de 7 bits que corresponde al código ASCII del caracter que está en
la punta del buffer. El buffer del teclado funciona como una cola.
Los caracteres se añaden a la cola cuando se teclea.
En cada ciclo de reloj, si KBD ENABLE está activa, el teclado elimina
el caracter que está en la punta de la cola.

Si en el momento que esta instrucción se ejecuta, KBD AVAILABLE está
desactivada, en Rd se almacena el valor $-1$.

Ensamblador: $kbd\hspace{5pt}R_s$

% \subsection*{PUSH RA}
%     Esta instrucción almacena en el tope de la pila el valor del
%     registro RA. La pila es simplemente el lugar de la memoria
%     al que señala el registro SP. Es un convenio común que la pila
%     ``crezca hacia abajo''. O sea, que SP empiece en un valor alto
%     y vaya decreciendo con cada PUSH y creciendo con cada POP.
%     Como los registros son de 32 bits, SP cambia de 4 en 4 con
%     cada PUSH y POP. Esta instrucción sería equivalente,
%     pero más rápida, que:

% \verb|    SUB SP,SP,4   | $\leftarrow$ \phantom{xx} Decrementa \verb|SP| en 4

% \verb|    MOV [SP],RA   | $\leftarrow$ \phantom{xx} Guarda \verb|RA| en \verb|RAM[SP]|

%     Codificación:

% \verb|    001000 .... xxxx ..................|

%     Ejemplo:

% \verb|    PUSH R9  | $\rightarrow$ \verb|  001000 0000 1001 000000000000000000|

% \subsection*{POP RD}
%     Esta instrucción almacena en RD el valor que está en el tope
%     de la pila y lo ``expulsa''. Realmente, basta con incrementar
%     SP en 4. Es equivalente, pero más rápida, que:

% \verb|    MOV RD,[SP]  | $\leftarrow$ \phantom{xx} Guarda en \verb|RA| el valor de \verb|RAM[SP]|

% \verb|    ADD SP,SP,4  | $\leftarrow$ \phantom{xx} Incrementa \verb|SP| en 2

%     Codificación:

% \verb|    001001 xxxx .... ..................|

%     Ejemplo:

% \verb|    POP R9  | $\rightarrow$ \verb|  001001 1001 0000 000000000000000000|



\section*{Program Counter}

  El valor del registro Program Counter (PC) representa la dirección de memoria de la
  próxima instrucción que se va a ejecutar. Como en S-MIPS las instrucciones
  ocupan 4 bytes, cada vez que el procesador ejecuta una instrucción que no sea
  de salto el valor de PC aumenta en 4.
  
  Las instrucciones de salto (beq, bne, blez, ...) suman su argumento (en complemento a 2) al de PC.
  El argumento indica el número de instrucciones a saltarse,por tanto es necesario multiplicar el 
  argumento por 4 antes de sumarlo al program counter.
  
  Como resultado de esto, si un programa quisiera saltarse una instrucción, debe
  hacer un salto con argumento 1:
  
  \verb|    add R3, R0, 46 |
  
  \verb|    add R4, R0, 46 |

  \verb|    beq R3, R4, 1 |

  \verb|    halt |

  En este ejemplo la instrucción halt simpre se salta. Otro resultado de este comportamiento 
  es que \verb|beq R0, R0, 0| es lo mismo que \verb|nop| y \verb|beq R0, R0, -1| es un ciclo infinito.

\section*{Acceso a la memoria}
  
  La memoria, para el programador de S-MIPS, es un array plano de 1MB,
  direccionable a nivel de 1 byte. La instrucción \textbf{lw} cuenta con 32 bits para direccionar
  la memoria RAM pero en el board utilizado solo hay instalados 1MB de memoria
  por tanto solo se van a utilizar para direccionar los valores en la RAM los 20 primeros bits de esta. 
  Nótese además que las transferencias entre la RAM y el procesador ocurren siempre en bloques de 4 bytes (32 bits).
  
  El procesador puede ignorar cualquier dirección de memoria que no sea divisible
  entre 4 e interpretarla como si fuera el número divisible por 4 más cercano por debajo. O sea las instrucciones
  \verb|lw R3, 0(R0)|, \verb|lw R3, 1(R0)|, \verb|lw R3, 2(R0)| y \verb|lw R3, 3(R0)| hacen lo mismo.
  
  Igualmente si un programa hace:
  
  \verb|    lw R3, 0(R0)|
  
  al registro R3 se copiarán los bytes 0, 1, 2 y 3 de la RAM. 
  
  S-MIPS es una arquitectura Little-Endian, que significa que, en cada
  palabra de 32 bits, el byte con dirección más pequeña es el menos significativo
  y el de dirección más grande es el más significativo. Esto luce ``al revés''
  de cómo se escriben normalmente los números.
  
  Supongamos que los primeros bytes de la RAM tienen estos valores:

\begin{center}
  \begin{minipage}[c]{2in}
    \begin{tabular}{*{7}{c}}
    0 & 1 & 2 & 3 & 4 & 5 &
    \end{tabular}

    \begin{tabular}{*{7}{|c}}
    \hline
    1 & 2 & 0 & 0 & 1 & 2 & $\cdots$ \\
    \hline
    \end{tabular} 
  \end{minipage}
\end{center}
  
  Al hacer
  
  \verb|    lw R3, 0(R0)|

  el byte 0 de la RAM, que tiene el valor 1 (00000001) va hacia la parte menos
  significative de R3 y el byte 3, con valor 0 (00000000), va hacia la más significativa.
  El valor final de R3 sería $2^9 + 2^0 = 513$
  
  Las escrituras se comportan de igual forma. Si se hace:
  
  \verb|    add R3, R0, 1027 |
  
  \verb|    sw  R3, 0(R0) |
  
  se escribirán para el byte 0 de la RAM la parte menos significativa del
  valor 1027 y para el byte 3 la parte más significativa.
  
  1027 es 00000000 00000000 00000100 00000011 en binario. El byte más significativo es 0
  y el menos significativo es 3 ($1027 = 2^{10} + 2^{1} + 2^{0}$). Así que la memoria
  quedaría así:
  
 \begin{center}
  \begin{minipage}[c]{2in}
    \begin{tabular}{*{7}{c}}
    0 & 1 & 2 & 3 & 4 & 5 &
    \end{tabular}

    \begin{tabular}{*{7}{|c}}
    \hline
    3 & 4 & 0 & 0 & 1 & 2 & $\cdots$ \\
    \hline
    \end{tabular} 
  \end{minipage}
\end{center}
    
\section*{Interfaz con la memoria}
  
  El módulo RAM del circuito S-MIPS Board implementa una memoria asíncrona de
  1MB. Esta memoria está organizada en 65536 bloques de 16 bytes. Esta RAM es más
  lenta que el procesador, y le toma varios ciclos leer y escribir datos. La
  cantidad de ciclos que toma hacer una lectura o una escritura están,
  respectivamente, en las salidas RT (Read Time) y WT (Write Time) que no cambian
  durtante la ejecución de un programa.
  
  (Aunque estas salidas sean constantes, no se deben usar sus valores ``a mano''
  dentro del CPU. Hay que obtenerlos de la RAM.  Los profesores vamos a probar
  distintas combinaciones de esos valores durante la verificación del procesador
  y este debe comportarse en concordancia.)
  
  Como la RAM tiene 65536 bloques, su entrada ADDR es de 16 bits.  Después de
  pasados RT ciclos de CPU de establecerse esa entrada, si la entrada CS (Chip
  Select) está en 1 y la entrada $\neg$R/W está en 0, la RAM proporcina los 16 bytes
  del bloque seleccionado mediante las 4 salidas de 32 bits O0, O1, O2 y O3.
  
  Análogamente, hay 4 entradas de 32 bits, I0, I1, I2 e I3, por las que se envían
  a la RAM valores para ser escritos en la dirección ADDR. Sin embargo, la
  escritura puede hacerse parcialmente, usando la entrada MASK.
  
  La RAM está dividida en 4 bancos que actúan como columnas o slices.  Cada
  bloque de 16 bytes de la RAM por tanto está dividido en 4 palabras de 4 bytes
  (32 bits). Los primeros 32 bytes de la RAM lucen así:
  \vspace{-0.5em}
  \begin{center}
  \begin{minipage}[c]{5.5in}
   \begin{multicols}{5}
    \centering
    Banco 0

    \begin{tabular}{|c|}
     \hline
     \verb| 00 01 02 03 | \\
     \verb| 16 17 16 19 | \\
     \verb| 32 33 34 35 | \\
     \verb| 48 49 50 51 | \\
     $\vdots$
    \end{tabular}
    \columnbreak

    Banco 1

    \begin{tabular}{|c|}
     \hline
     \verb| 04 05 06 07 | \\
     \verb| 20 21 22 23 | \\
     \verb| 36 37 38 39 | \\
     \verb| 52 53 54 55 | \\
     $\vdots$
    \end{tabular}
   \columnbreak

    Banco 2

    \begin{tabular}{|c|}
     \hline
     \verb| 08 09 10 11 | \\
     \verb| 24 25 26 27 | \\
     \verb| 40 41 42 43 | \\
     \verb| 56 57 58 59 | \\
     $\vdots$
    \end{tabular}
    \columnbreak

    Banco 3

    \begin{tabular}{|c|}
     \hline
     \verb| 12 13 14 15 | \\
     \verb| 28 29 30 31 | \\
     \verb| 44 45 46 47 | \\
     \verb| 60 61 62 63 | \\
     $\vdots$
    \end{tabular}
    \columnbreak

    \phantom{Banco X}
    \begin{tabular}{c}
     $\leftarrow$ Bloque 0 \\
     $\leftarrow$ Bloque 1 \\
     $\leftarrow$ Bloque 2 \\
     $\leftarrow$ Bloque 3 \\
     $\vdots$
    \end{tabular}

   \end{multicols}
  \end{minipage}
  \end{center}
  
  La entrada MASK es una entrada de 4 bits que selecciona cuál o cuales bancos
  van a ser modificados por una operación de escritura.  El bit menos
  significativo de MASK selecciona el Banco 0, el más significativo selecciona el
  Banco 3.
  
  Por ejemplo, si la entrada MASK tiene el valor 8 (en binario 1000) y la entrada
  ADDR es 0, la escritura solo afectaría los bytes 12, 13, 14 y 15 de la RAM, pues estos
  están en el bloque 0 y en el banco 3.  En esos 4 bytes se escribiría el valor
  de la entrada I3 (Data Input 3).
  
  Si la entrada MASK fuera 12 y ADDR fuera 2 (en binario 1100), la escritura
  modificaría los bytes 40, 41, 42, 43 y 44, 45, 46, 47 (bloque 2, bancos 2 y 3).  
  En esos 16 bytes se escribirían los valores de las entradas I2 e I3.
  
  Las entrada MASK no afecta las operaciones de lectura. Las salidas O0, O1, O2, O3
  siempre contienen el bloque completo solicitado en ADDR.
  
  Las escrituras se realizan cuando CS es 1 y $\neg$R/W es 1 y toman la cantidad de
  ciclos de CPU que indica la salida WT.  
  
\section*{Caché}
  
  La implementación del procesador S-MIPS debe incluir una caché de instrucciones y 
  una cache de datos para acelearar el proceso de lectura y/o escritura en la RAM. En ambos casos se puede
  aprovechar el hecho de que a partir de una sola lectura de la RAM se obtiene un bloque completo de 4 x 32bits.

  \textbf{Nota:} En caso de no implementar ningun tipo de caché la nota máxima a obtener es de 4ptos.
  
\section*{Costo y Presupuesto}
  
  El procesador va a tener un presupuesto máximo. Cada componente utilizada tiene un
  costo asociado y el costo total del procesador no puede tener un precio mayor de 100.
  Adjuto a este PDF encontrará un script \verb|price.py| que le permite calcular el costo de un circuito, 
  ejecute \verb|python price.py -h| para ver la ayuda. Los costos específicos de cada componente se pueden 
  consultar en el código fuente de este script.
  
\section*{Evaluación}
  
  El proyecto debe realizarse en equipos de no más de dos (2) estudiantes. El
  plazo para entregar los proyectos es el viernes 28 de abril de 2017. Después de
  la entrega, se realizará una revisión oral con un profesor y los miembros de
  cada equipo.

  Se puede entregar y revisarse en cualquier momento antes de la expiración del plazo.

  Pasar todos los casos de prueba es un requisito para realizar la revisión oral.
  El proyecto no se aprueba si el circuito no ha pasado todas la pruebas antes de o en el día de la entrega. 
  Ninguno de los casos de prueba obligatorios comprueban la existencia y/o funcinamiento de una memoria caché.

  La nota del proyecto se decide en la revisión oral. 

\section*{Cambios}

\subsection*{v1.1}
  \begin{itemize}
    \item Se cambió el opcode de la instrucción \textbf{bgtz} y \textbf{bltz} para evitar colisión con \textbf{blez}. 
    \item El script \textit{price.py} para el cálculo del precio del procesador devuelve un precio 1000 veces menor.
    \item Bajó el costo máximo permitido de 1000 unidades a 100.
    \item El registro destino de las instrucciones \textbf{rnd} y \textbf{kbd} ahora es Rd (era Rs).
  \end{itemize}

\end{document}
